\documentclass[12pt,a4paper]{book}
\usepackage{asymptote}
\usepackage{times}
\usepackage[margin=2.5cm]{geometry}
\usepackage{tabularx}
\usepackage[lite]{mtpro2}
\usepackage{expl3}
\ExplSyntaxOn
% make an internal function available to the user
\cs_set_eq:NN \fpeval \fp_eval:n
\ExplSyntaxOff
\def\asydir{asy}
%latexmk main.tex -ps -pdf- -dvi- -pvc
%live-server www.html in folder//npm install -g live-server ///
%while true; do inotifywait -e CLOSE_WRITE www.asy; asy -f html  www.asy; done; ///

%%%%%%%%%%%%%%%%%%%%%%%%%%%%%%%%%%%%%%%%%%
%https://github.com/fissart/artbook/raw/main/artbook.pdf
\usepackage[spanish]{babel}
\usepackage[centertags]{amsmath}
\usepackage{amsfonts}
\usepackage{pstricks-add}
\usepackage{graphicx}
\usepackage{longtable}
\usepackage{booktabs}
%\usepackage{ulem}
%\usepackage{textcomp}
%\usepackage{showframe}
%\usepackage[utf8]{inputenc}
%\usepackage{hyperref}
%%%%%%%%%%%%%%%%%%%%%%%%%%%%%%%%%%%%%%%%
\usepackage[apaciteclassic, nosectionbib, tocbib]{apacite}
\usepackage{usebib}
\bibinput{bb}
\usepackage{makeidx}
\makeindex
%%%%%%%%%%%%%%%%%%%%%%%%%%%%%%%%%%%%%%%%
\newtheorem{comen}{Comentario}[chapter]
\newtheorem{thm}{Teorema}[chapter]
\newtheorem{defn}[thm]{Definición}
\newtheorem{lem}{Lema}[thm]
\newtheorem{cor}{Corolario}[thm]
\newtheorem{prop}{Proposicion}[thm]
\newtheorem{rem}{Remark}[thm]
\newtheorem{ill}{Illustration}[thm]
%%%%%%%%%%%%%%%%%%%%%%%%%%%%%%%%%%%%%%%%

\newcommand{\qw}{\phi}
\newcommand{\Real}{\mathbb R}
\newcommand{\pa}[1]{\left(#1\right)}
%%%%%%%%%%%%%%%%%%%%%%%%%%%%%%%%%%%%%%%%
%%%%%%%%%%%%%%%%%%%%%%%%%%%%%%%%%%%%%%%%
%%%%%%%%%%%%%%%%%%%%%%%%%%%%%%%%%%%%%%%%


\begin{document}
%%%%%%%%%%%%%%%%%%%%%%%%%%%%%%%%%%%%%%%%
\begin{asydef}
	settings.prc=false;
	texpreamble("\usepackage[lite,subscriptcorrection,slantedGreek,nofontinfo,amsbb,eucal]{mtpro2}");
	defaultpen(fontsize(12 pt));
	defaultpen(linewidth(0.7pt));
	//settings.render=1;
\end{asydef}
%%%%%%%%%%%%%%%%%%%%%%%%%%%%%%%%%%%%%%%%
%%%%%%%%%%%%%%%%%%%%%%%%%%%%%%%%%%%%%%%%
\thispagestyle{empty}
{
	\centering
	\vspace{3cm}
	\bf{\huge ARTE Y MATEMÁTICAS}\\
	\bf{\large ARTE Y MATEMÁTICAS}\\
	\vspace{0.5cm}
	\bf{RICARDO}\\
	\vspace{.5cm}

	\begin{asy}
	import graph;

size(200,150,IgnoreAspect);

real[] x={0,1,2,3};
real[] y=x^2;

draw(graph(x,y),red);

xaxis("$x$",BottomTop,LeftTicks);
yaxis("$y$",LeftRight,
      RightTicks(Label(fontsize(8pt)),new real[]{0,4,9}));
	\end{asy}
	\vfill
	%ps://asy.marris.fr/
	Departmento de mathemática y física, FIMGC USNCH\\
	\emph{E-mail}: \texttt{ricardomallqui@gmail.com}\\
	URL: \textsf{www.fractales.com}

}
\newpage
%%%%%%%%%%%%%%%%%%%%%%%%%%%%%%%%%%%%%%%%

{
	\thispagestyle{empty}
	\noindent\bf{Arte y Matemáticas}\\
	\bf{Ricardo Michel Mallqui Baños}\\
	\vspace{3cm}

	\noindent Un libro basado en código asymptote LaTeX y pstricks\\

	\noindent Bibliografia.\\
	\noindent Incluye Indice.\\
	1. Geometría, Diferencial. 2. Curvas. 3. Superficies. \\
	\vfill
	\noindent
	\begin{asy}
		size(300);
		path ltrans(path p,int d)
		{
				path a=rotate(65)*scale(0.4)*p;
				return shift(point(p,(1/d)*length(p))-point(a,0))*a;
			}
		path rtrans(path p, int d)
		{
				path a=reflect(point(p,0),point(p,length(p)))*rotate(65)*scale(0.35)*p;
				return shift(point(p,(1/d)*length(p))-point(a,0))*a;
			}

		void drawtree(int depth, path branch)
		{
				if(depth == 0) return;
				real breakp=(1/depth)*length(branch);
				draw(subpath(branch,0,breakp),blue);
				drawtree(depth-1,subpath(branch,breakp,length(branch)));
				drawtree(depth-1,ltrans(branch,depth));
				drawtree(depth-1,rtrans(branch,depth));
				return;
			}

		path start=(0,0)..controls (-1/10,1/3) and (-1/20,2/3)..(1/20,1);
		drawtree(7,start);
	\end{asy}

	%\noindent \texttt{\textregistered\;\textcopyright\; 2023 pa-esfa, Inc. UNSCH, Huamanga}\\
	\noindent %\texttt{\textregistered\;\textcopyright}\\
	\texttt{Todos los derechos reservados. Ninguna parte de esto
		libro puede ser reproducido en cualquier forma,
		o por cualquier medio, sin permiso
		por escrito del editor.}\\
	Departmento de mathemática y física, FIMGC USNCH\\
	\emph{E-mail}: \texttt{ricardomallqui@gmail.com}\\
	URL: \textsf{www.fractales.com}

}
%%%%%%%%%%%%%%%%%%%%%%%%%%%%%%%%%%%%%%%%
\newpage
\renewcommand\listfigurename{Índice general}
\pagenumbering{roman}
\setcounter{page}{1}
\addcontentsline{toc}{chapter}{Índice general}
\tableofcontents
\renewcommand\listfigurename{Lista de figuras}
\addcontentsline{toc}{chapter}{Lista de figuras}
\listoffigures

\renewcommand\listtablename{Lista de tablas}
\addcontentsline{toc}{chapter}{Lista de tablas}
\listoftables
%\newpage
%\clearpage
%%%%%%%%%%%%%%%%%%%%%%%%%%%%%%%%%%%%%%%%

\chapter*{Resumen}
\addcontentsline{toc}{chapter}{Resumen}

wwwwww


\chapter*{Presentación}
\addcontentsline{toc}{chapter}{Presentación}

%\underline{\underline{Double underlined text}}
%{Double underlined text}
%\textsl{\underline{Slanted underlined}}
%\textsc{\underline{Small caps underlined}}

Matemáticas en el arte plástico nace del intento de poner en orden, la  noción intuitiva que se tiene sobre la estructura compositiva en el arte plástico y hacerla un tanto rigurosa en un aspecto lógico de formas, sobre una
base estructural geométrica.

El número áureo, es uno de los fractales más interesantes, el objetivo es
hacer reconocer, de que modo, este número esta relacionado con los fracta-
les y generalizarlo, a conceptos mucho más elaborados, para poder aplicar-
las en el arte plástico. El universo tiene un lenguaje, basado en los números,
que la describe casi por completo, lo cual implica que está presente, en todos
los fenómenos de la realidad.

Se sabe que a pesar de lo discutible de su conocimiento sobre el número
áureo, Platón se ocupa de estudiar el origen y la estructura del cosmos,
caso que intentó, usando los cinco sólidos platónicos, Para Platón los sólidos
corresponden a una de las partículas que conformaban cada uno de los
elementos es decir la tierra lo asocia con el cubo, el fuego con el tetraedro,
el aire con el octaedro, el agua con el icosaedro y finalmente el universo,
como un todo asociado con el dodecaedro las cuales se tratan en el Capítulo 3.

Se analizo el libro del teólogo y matemático Lucca Paccioli que trata
sore la sección áurea en base al legado dejado por Platón y Euclides, en
su libro La Divina Proporción donde describe la construcción de los cinco
sólidos platónicos, el nombre Platónico debido la descripción constructiva
de estos sólidos por Platón, asociados a la estética, la mística, la cósmica
y la teológica, que conmovió a todas las generaciones, desde los pueblos
neolíticos hasta nuestros dias.

Lucca Pacioli publica su libro La Divina proporción en 1509, donde plan-
tea cinco razones por la que estima apropiado considerar divino al número
de oro, primero la unicidad del número de oro, la compara con la unici-
dad de dios, segundo el hecho de que esté definido con tres segmentos de
recta lo relaciona con la trinidad, tercero la inconmensurabilidad del nú-
mero de oro y la inconmensurabilidad de Dios son equivalentes, cuarto la
utosimilitud asociada al número de oro lo compara con la omnipotencia e
invariabilidad, finalmente el quinto, de la misma manera en que Dios dio ser
al universo a través de la quinta esencia, representada por el dodecaedro,
el número de oro, dio ser al dodecaedro.

Pero si bien ejemplos y contraejemplos constituyen una trascendencia
importante, en algún proceso, se trato de evitar que el lector, se quede con
la idea de que los números están trivialmente en alguna aplicación, por
ello se ha procurado presentar de manera ordenada en el cuerpo básico del
texto, de manera que exista una secuencia de conceptos implicados unos
con otros.

Cinco capítulos son los que forman el libro, el primero sobre la sección
áurea, el segundo sobre formas geométricas en el plano, el tercero sobre los
sólidos platónicos, el cuarto sobre los fractales, el quinto sobre los principios
de la composición plástica, el sexto sobre superficies esto con el objetivo de
establecer algunos términos en la escultura y reconocer sus propiedades
para ser aplicada adecuadamente y finalmente un pequeño apéndice.

Cinco capítulos son los que forman el libro, el primero sobre la sección
áurea, el segundo sobre formas geométricas en el plano, el tercero sobre los
sólidos platónicos, el cuarto sobre los fractales, el quinto sobre los principios
de la composición plástica, el sexto sobre superficies esto con el objetivo de
establecer algunos términos en la escultura y reconocer sus propiedades
para ser aplicada adecuadamente y finalmente un pequeño apéndice.
\chapter{First}
\chapter{First}
\chapter{First}
\chapter{First}


Resumen ejecutivo: Un resumen de las conclusiones principales del informe. 
Marco teórico: Una contextualización del tema y los conceptos clave. 
Marco metodológico: Explica cómo se realizó la investigación o análisis. 
Resultados: Presenta los datos y hallazgos de la investigación o análisis. 
Discusión: Analiza e interpreta los resultados y los relaciona con la literatura existente. 
Recomendaciones: Sugiere acciones o medidas a tomar en base a los resultados. 
Glosario de términos: Define conceptos clave utilizados en el informe. 


\bibliographystyle{apacite}
\bibliography{bb}
\addcontentsline{toc}{chapter}{Indices}
\printindex

\appendix
\pagenumbering{roman}
\setcounter{page}{1}
\chapter{www}
\chapter{www}

\end{document}

